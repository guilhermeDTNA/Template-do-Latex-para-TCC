\documentclass[xcolor=table]{beamer}
% \usepackage[utf8]{inputenc}
\usepackage{default}
\mode<presentation>{
    %\usepackage{beamerthemeshadow}
    %\usepackage{beamerthemeshadow}
    %\usepackage{beamerthemePaloAlto}
    %\usepackage{beamerthemeSzeged}
    %\usepackage{beamerthemeAntibes}
    %\usepackage{beamerthemeBergen}
    %\usepackage{beamerthemeBerkeley}
    %\usepackage{beamerthemeBerlin}
    %\usepackage{beamerthemeBoadilla}  
    %\usepackage{beamerthemeboxes}  
    %\usepackage{beamerthemeCambridgeUS}
    %\usepackage{beamerthemeCopenhagen} 
    %\usepackage{beamerthemeDarmstadt}
    %\usepackage{beamerthemeDresden}
    %\usepackage{beamerthemeFrankfurt}
    %\usepackage{beamerthemeGoettingen}
    %\usepackage{beamerthemeHannover}
    \usepackage{beamerthemeIlmenau}
    %\usepackage{beamerthemeJuanLesPins} 
    %\usepackage{beamerthemeLuebeck} 
    \usepackage{pgf,pgfarrows,pgfnodes,pgfautomata,pgfheaps,pgfshade}
    \usepackage{clrscode3e}
    
%     \usepackage[table,xcdraw]{xcolor}

    \beamertemplatetransparentcovereddynamic
    \beamertemplateballitem
    %\beamertemplatefootpagenumber
}
\mode<handout>{
    \usepackage[bar]{beamerthemetree}
    % Colocando um fundo cinza quando for gerar transparências para serem impressas
    % mais de uma transparência por página
    \beamertemplatesolidbackgroundcolor{black!5}
}
% \usepackage[ruled,chapter]{algorithm}
\usepackage{algorithm}	
\usepackage{algorithmic}

\usepackage{comment}
\usepackage{amsmath,amssymb}
\usepackage[brazil]{varioref}
\usepackage[english,brazil]{babel}
\usepackage[utf8]{inputenc}
\usepackage{graphicx}
\usepackage[alf]{abntex2cite}
\usepackage{abntex2abrev}
% \usepackage[round,sort,authoryear,nonamebreak]{natbib}
% \usepackage[authoryear]{natbib}
%\usepackage{minitoc-hyperref}
%\usepackage{listings}
%\usepackage{listings}
%\usepackage{colortbl}
%\usepackage{pslatex}
\beamertemplatetransparentcovereddynamic

\usepackage{tikz}
\usetikzlibrary{arrows}
\tikzstyle{block}=[draw opacity=0.7,line width=1.4cm]

\newcommand*\oldmacro{}%
\let\oldmacro\insertshorttitle%
\renewcommand*\insertshorttitle{%
  \oldmacro\hfill%
  \insertframenumber\,/\,\inserttotalframenumber}

\title[Trabalho de Conclusão de Curso]{Trabalho de Conclusão de Curso}
\subtitle{Título do trabalho}

%\titlegraphic{}

\author[Nome do Autor]{Nome do Autor}
  \institute[UFVJM]{Universidade Federal dos Vales do Jequitinhonha e Mucuri \newline
  Bacharelado em Sistemas de Informação \newline
%   \inst{Departamento de Computação
	  
     Orientador: Prof. Fulano\\
     Coorientador: Cicrano\\
     $~$\\
%      Disciplina: Nome da Disciplina\\
}
\logo{\includegraphics[scale=0.1]{logo-ufvjm.jpg}}
% Se comentar a linha abaixo, irá aparecer a data quando foi compilada a apresentação
%\date{\textcolor{red}{III Encontro Mineiro de Equações Diferenciais, 2009}}
%\pgfdeclareimage[height=0.4cm]{das}{figs/logodas}
%\pgfdeclareimage[height=1cm]{logo}{logos}
% pode-se colocar o LOGO assim
%\logo{\pgfuseimage{logo}}
% ou...
%\logo{\vbox{\hbox to 3cm{\hfil\pgfuseimage{logo}}}}

\begin{document}
%\xdefinecolor{MyGreen}{rgb}{0,0.6,0}
%\beamerboxesdeclarecolorscheme{MyGreen}{MyGreen}{MyGreen!20!averagebackgroundcolor}
\beamerboxesdeclarecolorscheme{formula}{white}{blue!250!averagebackgroundcolor}
\frame{\titlepage}

%\frame{\titlepage}
\part{Presentation}
% \section{Sumário}

\frame{
\frametitle{Sumário}
\tableofcontents
}

\AtBeginSection[]{
  \frame<handout:0>{
    %\frametitle{Sumário}
    \tableofcontents[current,currentsection]
  }
}

%===================================Slide=================================================

\section{Introdução}


%Inserindo a justificativa 
\begin{frame}
    \frametitle{Motivação}
    
    \textbf{Problemas encontrados:}
    \begin{itemize}
        \item Nenhuma
    \end{itemize}

\end{frame}


\begin{frame}
    %Insira os objetivos
    \frametitle{Objetivos}
    \textbf{Objetivo geral:} Acabar com o projeto antes que ele acabe comigo.
    
    \textbf{Objetivos específicos:}
    \begin{itemize}
        \item Criar; e
        \item Apresentar.
    \end{itemize}
\end{frame}


\section{Desenvolvimento}

\begin{frame}
    %Inserindo tabelas
    \frametitle{Inserção de tabelas}
    
    % Please add the following required packages to your document preamble:
    % \usepackage[table,xcdraw]{xcolor}
    % If you use beamer only pass "xcolor=table" option, i.e. \documentclass[xcolor=table]{beamer}
    \begin{table}[h]
    \centering
    \caption{Título da tabela. Fonte: Autor.}
    \resizebox{8cm}{!}{ %coloca toda a tabela dentro do slide
        \begin{tabular}{l|c|c}
        \multicolumn{1}{c|}{\textbf{Coluna 1}}  & \textit{\textbf{Coluna 2}} & \textit{\textbf{Coluna 3}} \\
        \hline 
        Linha 1      & 83,33\%                 & 68,14\%                     \\
        LInha 2   & 51,85\%                 & 66,67\%                     \\
        
        \hline 
        \end{tabular}
    }
    \label{comaparacao-cms}
    \end{table}
\end{frame}


\section{Resultados}
\begin{frame}
    %Inserindo imagens
    \frametitle{Apresentação das Telas: Exemplo de inserção de imagem}
    \begin{figure}[htb]
        \centering
        \includegraphics[width=0.4\textwidth]{figuras/tux.png}
        \label{fig:login-portal}
    \end{figure}
\end{frame}


\section{Discussão}
\begin{frame}
    %Insira a discussão dos resultados
    \frametitle{Discussão}
    \begin{itemize}
        \item Cumprimento de todos os objetivos
        
    \end{itemize}
\end{frame}


\begin{frame}
    %Insira os trabalhos futuros
    \frametitle{Dificuldades Encontradas}
    \begin{itemize}
     \item Nenhuma
    \end{itemize}

\end{frame}



\section{Trabalhos futuros}
\begin{frame}
    %Insira os trabalhos futuros
    \frametitle{Trabalhos futuros}
    \begin{itemize}
        \item Nenhum;

        \end{itemize}

    
\end{frame}

    

\section{Referências Bibliográficas}

\begin{frame}
    %Insira as referências manualmente (Ordem alfabética)
    \frametitle{Referências Bibliográficas}
    \scriptsize{
        Leite, G. R. Desenvolvimento de um sistema web para a Associação de Protetores da Bacia Hidrográfica do Rio Gorutuba ``Kuruatuba'', Diamantina, MG, Brasil, 2021. \newline
        
}
\end{frame}

\begin{frame}
    %Insira as referências manualmente (Ordem alfabética)
    \frametitle{Referências Bibliográficas}
    \scriptsize{

        Leite, R. R. Análise da Busca Local Multiobjetivo Baseada em Indicador na resolução do Problema de Roteamento de Veículos com serviços de entrega obrigatória e coleta, Diamantina, MG, Brasil, 2016. \newline
    }
\end{frame}


\end{document}
